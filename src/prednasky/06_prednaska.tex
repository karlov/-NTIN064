\section{\texorpdfstring{G\"{o}delovy věty}{G\"{o}delovy věty}}
\vspace{5mm}
\large

\begin{definition}[Rozumná teorie]
	Rozumná teorie musí být:
	\begin{itemize}
		\item bezesporná
		\item axiomatizovatelná
		\item Základní aritmetické síly (adekvátní)
	\end{itemize}
\end{definition}

\begin{definition}[Axiomatizovatelná teorie]
	Axiomatizovatelná teorie je právě když množina dokazatelných formulí je r.s.
\end{definition}

Způsoby řešení paradoxu v teorii množin:
\begin{itemize}
	\item intuicionisty/kontruktivisty finitisty - odmítají nekonečno.
		Jde vybudovat spoustu věcí, ale vznikající teorie je kostrbatá a nepříjemná.

		Již Bolzano upozorňoval, že nekonečno je nevlastní pojem, překročující lidskou existenci. Viz ~\cite{bolzano2014paradoxes}
	\item Hilbertův formalismus - vybudovat teorie z logiky 1. řádu a aby každé tvrzení šlo jednoznačně dokázat nebo vyvrátit + aby teorie byla bezesporná.
\end{itemize}

% todo Peano aritmetika

\begin{note}
	Tzv. \href{https://en.wikipedia.org/wiki/Presburger_arithmetic}{Presburgerova Aritmetika}  bez násobení je rekurzivní a konzistentní.
\end{note}

\begin{note}
	Teorie vyčíslitelnosti dává ekvivalentní pohled na G\"{o}delovy věty. Které původně byly vyjádřené přes jazyk aritmetiky.
\end{note}

\begin{definition}[Reprezentovatelnost]
	$f \in ČRF$ je reprezentovatelná v teorii $T$ pokud existuje formule $F$:
	\[ f(x) = y \Rightarrow \vdash_T F(\overline{x}, \overline{y}) \]
	\[ \vdash_T F(x, y) \land F(x, z) \Rightarrow y = x\ \text{(funkční vlastnost, aby nebyla jen relace)} \]

	Pokud platí obě podminky a $T$ je bezesporná, tak
	\[ \{ (x, y), \vdash_T F(\overline{x}, \overline{y}) \}\ \text{graf nějaké funkce} \supseteq f \]
\end{definition}

\begin{observation}
	ČRF jsou reprezentovatelné v libovolné teorii ZAS.

	Víme, že pro $\varphi \in ČRF$ graf
	\[ \{ (x, y): \varphi(x) \simeq y \} \]
	je r.s.

	Z důsledku RDPM věty \cref{rdpm_am} máme
	\[ \text{r.s.}\ \approx \exists(p_1(\ldots) = p_2(\ldots)) \]

	Rovnost dvou polynomů je jednodušé formalizovatelná v teorii ZAS.
\end{observation}

\begin{theorem}[Vztah $\N a\ T$]
	% todo Sigma_1 formule z earitmeticke hierarchie, add link
	Pak pokud máme $\Sigma_1$ formule v jazyce ZAS, které jsou pravdivé v $\N$ jsou v T (ZAS) dokazatelné.
	\[ \N \models \exists (\ldots) \Rightarrow \vdash_T \exists (\ldots) \]
\end{theorem}
\begin{proof}
	Vezmeme např následující formule:
	\[ \exists x (x + \overline{7} = \overline{17})\]
	Od teorie $T$ chceme, aby formuli ověřila.

	Pokud v $\N$ je pravdivá nějaká $\Sigma_1$ formule, tak existuje tzv. $\Sigma_1$ svědek.
	Což je jedno nebo několik přirozených čísel které splňují formuli po dosazení.
	Teorie zkontroluje rovnost termu.
\end{proof}

\begin{lemma}[Disjunktní množiny formule(BD)]
	Nechť $T$ je bezesporná, ZAS.
	Pokud $A, B$ jsou disjunktní r.s. tak existuje $\Sigma_1$-formule $G$:

	\[ x \in A \Rightarrow \vdash_T G(\overline{x}) \]
	\[ x \in B \Rightarrow \vdash_T \neg G(\overline{x}) \]
\end{lemma}
\begin{proof}
	Sestavíme formuli:
	\[ \twopartdef {\varphi(x) = 0}{x \in A}{\varphi(x) = 1}{x \in B} \]

	Pak
	\[ x \in A \Rightarrow \vdash_T F(\overline{x}, \overline{0}) \]
	\[ x \in B \Rightarrow \vdash_T F(\overline{x}, \overline{1}) \]

	Z bezespornosti
	\[ x \in B \Rightarrow \vdash_T F(\overline{x}, \overline{1}) \Rightarrow \vdash_T \neg F(\overline{x}, \overline{0}) \]

	Pak hledaná formule je
	\[ G(x) = F(x, \overline{0}) \]
\end{proof}

\begin{theorem}[G\"{o}delovy věty]\label{godel}
	Jestliže teorie $T$ 1. řádu má základní aritmetickou sílu a je bezesporná, pak:
	\begin{enumerate}
		\item množina dokazatelných v $T$ formulí není rekurzivní
		\item pokud je $T$ navíc axiomatizovatelná, tak existuje uzavřená formule (sentence) $F$ taková, že:
			\[ T \nvdash F \land T \nvdash \neg F \]
		\item (2. Věta) axiomatizovatelnost + Indukce $\Sigma_1$ (stačí i trochu méně) tak v $T$ nelze dokázat její bezespornost (consistency).
		Formálně:
		\[ T \nvdash Con_T \]
		kde $Con_T$ je formule vyjádřující konsistence, např
		\[ \neg \exists \text{proof}\ (\overline{0} = \overline{1}) \]
	\end{enumerate}
\end{theorem}
\begin{proof}~
	\begin{enumerate}
		\item Nechť $T$ ZAS, bezesporná. Nechť $(A, B)$ r.s. efektivně neoddělitelné.
			\[ A_1 = \{ x: \vdash_T G(\overline{x}) \} \]
			\[ B_1 = \{ x: \vdash_T \neg G(\overline{x}) \} \]
			$A \subseteq A_1$ je r.s. obal, podobně $B \subseteq B_1$.

			Jelikož $A, B$ jsou efektivně neoddělitelné $\Rightarrow$ jsou rekurzivně neoddělitelné.
			Z toho $A_1, B_1$ nejsou rekurzivní.
			Kdyby byly rekurzivní, tak by každá z nich rekurzivně oddělovala $(A,B)$.

			%Q: protože jinak kdyby byly rekurzivní, tak by nevyplnili celý prostor?
		\item Když přidáme axiomatizovatelnost, tak $A_1, B_1$  jsou r.s.

			Pak z efektivní neoddělitelnosti efektivně najdeme takové $k \notin A_1 \cup B_1$ že:
			\[ \nvdash_T G(\overline{k}) \land \vdash_T \neg G(\overline{k}) \]
		\item BD, formalizace, hodně logiky.
	\end{enumerate}

	Jinými slovy, máme následující:
	\[ \exists \text{můj důkaz IF existuje kratší důkaz mé negace} \]
	A symetricky pro gace. Nedochází k žádnému paradoxu, oproti paradoxu lháře.
\end{proof}

\subsection{Kalibrace síly teorie}

\begin{note}
	Konečná verze Ramsey věty je v PA nedokazatelná.
\end{note}

\begin{note}
	PA má sílu
	\[ \varepsilon_0 = \omega^{\omega^{\ldots}} \]
	kde exponent je dlouhý $\omega$.

	%Což je největší ordinál, který ještě dává dobré uspořádaní.

	Zkoumá tuto oblast \emph{proof theory}.
\end{note}

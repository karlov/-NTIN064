\section{\texorpdfstring{Věty o rekurzi}{Věty o rekurzi}}
\vspace{5mm}
\large

Taky se jmenuji věty o pevném bodě. Používá se self-refrenční trik

\begin{theorem}[O rekurzi 1]\label{rek_1}
	Pokud $f$ je ČRF (pro jednoduchost 1 proměnné) $\Rightarrow$ (efektivně z indexů)
	\[ \exists a\ \forall x: \varphi_a(x) \simeq \varphi_{f(a)} (x) \]

	Jinými slovy: pokud $f(a) \downarrow \Rightarrow \varphi_a$ a $\varphi_{f(a)}$ jsou stejné funkce.
	Programy nejsou stejné, ale vyčísluji stejnou funkci.

	Pokud ale $f(a) \uparrow \Rightarrow \forall x: \varphi_a(x) \uparrow$.
\end{theorem}
\begin{proof}
	\[ \varphi_{f(s_1(z, z))} (x) \simeq \Psi_2(e, z, n) \]
	Protože levá strana je efektivně vyčíslitelná. $e$ je program který počítá levou stranu.

	Pak dle s-m-n věty \cref{s_m_n}
	\[ \varphi_{f(s_1(z, z))} (x) \simeq \Psi_2(e, z, x) \simeq \varphi_{s_1(e, z)} (x) \]
	polož $z = e$, dostaneme
	\[ \varphi_{f(s_1(e, e))} (x) \simeq \varphi_{s_1(e, e)} (x) \]
	Neboli
	\[ a = s_1(e, e) \]


	Který program počítá déle?

	Program $e = f(s_1(z, z))$:
	\begin{enumerate}
		\item spočítej $s_1(z, z)$.
		\item spočítej $f(s_1(z, z))$.\\
			který ale nemusí konvergovat
		\item $if(f(s_1(z, z)) \downarrow)$\\
			spust $e$ na vstup $x$.
	\end{enumerate}

	Program $a = s_1(z, z)$:
	\begin{enumerate}
		\item dostane $x$ na vstupu, kvůli s-m-n přidá $e$ ke vstupu
		\item spustí program $e$ na vstup $\langle e, x \rangle$.
		\item spočítej $a = s_1(z, z)$.\\
			Tady spočítal svůj vlastní index.
		\item spočítej $f(s_1(z, z))$ tzn $f(a)$.\\
			který ale nemusí konvergovat
		\item $if(f(s_1(z, z)) \downarrow)$ then \\
			spust $f(a)$ na vstup $x$.
	\end{enumerate}

	Takže $a$ počítá déle.
\end{proof}

\begin{theorem}[O rekurzi 2]
	Pokud $f$ je ČRF $(n + 1)$ proměnných $\Rightarrow$ ORF (dokonce PRF)
	\[ \varphi_{h(y_1, \ldots, y_n)}(x) \simeq \varphi_{f(h(y_1, \ldots, y_n, y_1, \ldots, y_n))} (x) \]

	Pokud smažeme $y$-ny, tak dostaneme právě Větu o rekurzi 1 \cref{rek_1}.

	Pevné body efektivně na parametrech.
\end{theorem}
\begin{proof}
	Analogicky jako důkaz Věty o rekurzi 1 \cref{rek_1}.
	Jenom aplikujeme s-m-n \cref{s_m_n} na větší počet parametrů.

	\[ \varphi_{f(s_{n + 1}(z, z, y_1, \ldots, y_n), y_1, \ldots, y_n)} (x) \simeq \Psi_{n + 2}(e, z, y_1, \ldots, y_n, x) \simeq \varphi_{s_{n+1}(e, z, y_1, \ldots, y_n)} (x) \]
	Pak
	\[ h(y_1, \ldots, y_n) = s_{n+1}(e, e, y_1, \ldots, y_n) \]
\end{proof}

\begin{theorem}[O rekurzi $\infty$]
	Pokud $f$ je ČRF $\Rightarrow \exists$ prostá ORF g (dokonce PRF)
	\[ \varphi_{g(j)}(x) \simeq \varphi_{f(g(j))} (x) \]

	Pak pevných bodů je nekonečno
	\[ g(0), g(1), \ldots \]
\end{theorem}
\begin{proof}
	\[ \varphi_{f(s_2(z, z, j))} (x) \simeq \Psi_2(e, z, j, x) \simeq \varphi_{s_2(e, z, j)} (x) \]
	Zvolme
	\[ g(j) = s_2(e, e, j) \]
\end{proof}

\begin{theorem}[O rekurzi 3]
	Pokud $h(x, z_1, \ldots, z_n)$ je ČRF, pak existuje $a \in \N$ t.ž. $a$ je indexem funkce
	\[ h(a, z_1, \ldots, z_n) \]
\end{theorem}
\begin{proof}
	\[ h(x, z_1, \ldots, z_n) \simeq \Psi_{n + 1}(e, x, z_1, \ldots, z_n) \simeq \varphi_{s_1(e, x)} (z_1, \ldots, z_n) \]

	Pak aplikujeme Větu o rekurzi \cref{rek_1} na funkci $s_1(e, x)$.
\end{proof}

\begin{note}
	Věty o rekurzi platí nejen pro ČRF ale taky pro jejich definiční obory.
	Takže
	\[ f \in ČRF\ \Rightarrow \exists a: W_a = W_{f(a)} \]
\end{note}

\begin{theorem}[Program vlastní kod]
	Existuje $n_0$:
	\[ \varphi_{n_0} (n_0) = n_0 \]
	Program který vypíše svůj vlastní kód.
\end{theorem}
\begin{proof}
	Pořídíme si pomocnou ORF
	\[ \alpha(n, w) = n \]
	použijeme s-m-n \cref{s_m_n}
	\[ \alpha(n, w) \simeq \varphi_{f(n)} (w) \]
	Kde
	\[ \alpha(n, w) \simeq \Psi(e, n, w) \simeq \varphi_{s_1(e, n)} (w) \]
	Stačí použít Větu o rekurzi \cref{rek_1} k $f$
	\[ \varphi_{n_0} (n) \simeq \varphi_{f(n_0)} (w) = n_0 \]
\end{proof}

\begin{theorem}[Rekurze v indexech (BD)]
	Existuje ORF $f$:
	\[ W_{f(y)} = \{ f(0), \ldots, f(y - 1) \} \]
\end{theorem}
\begin{proof}
	Hint: hledáme index $f$ kterých je spočetně mnoho.
	Musíme použít Větu o rekurzi na úrovně indexů.
\end{proof}

\begin{theorem}[Rice podruhe]
	Pokud $F$ je netriviální třída ČRF (nebo r.s množin).
	Není prázdná, nebo má všechny.
	Pak indexová množina
	\[ A_F = \{ x |\ \varphi_x \in F \} \]
	Je nerekurzivní.
\end{theorem}
\begin{proof}
	Z netriviálity F
	\[ \exists a \in A \land b \in \overline{A} \]
	Nechť sporem $A$ je rekurzivní.

	Uděláme funkci $h$ t.ž
	\begin{gather*}
		\forall x \in A: h(x) = b \\
		\forall y \in \overline{A}: h(y) = a
	\end{gather*}

	Protože $A$ je rekurzivní, tak $h$ je ORF $\Rightarrow$ existuje pevný bod třeba $n_0$
	\[ n_0 \in A \Rightarrow h(n_0) \in \overline{A} \]

	Z Věty o rekurzi ale víme
	\[ \varphi_{n_0} = \varphi_{h(n_0)} \]
	takže $n_0, h(n_0)$ musí být ve stejné množině.

	Z toho $A$ nené rekurzivní.
\end{proof}

\begin{theorem}[O rekurzi (BD)]
	Nechť $f \in ČRF$ pak
	//TODO
\end{theorem}
\begin{proof}
	\[ \varphi_{\varphi_u(u)} (z) \simeq \Psi_2(a, u, z) \stackrel{\text{s-m-n}}{\simeq} \varphi_{s_1(a, u)} (z) \simeq \varphi_{d(u)} (z) \simeq \varphi_{\varphi_e(u)} (z) \]
	Všimneme si, že
	\[ \varphi_u(x) \]
	je \emph{matice} funkci. Implikace nahoře ukazuje, že její diagonála $\varphi_u(u)$ se rovná řádku $\varphi_e$.

	Ukážeme, že $f$ permutuje řádky.
	\[ \varphi_{f \circ \varphi_u (x)} (z) \simeq \varphi_{\beta(u, x)} \stackrel{\text{s-m-n}}{\simeq} \varphi_{\varphi_{H(u)}(x)} (z) \]
	Kde
	\[ H(u) = s_1(b, u) \]
	Z toho u-tý řádek se zobrazí na $H(u)$-tý.
	Speciálně
	\[ e \to H(e) \]
	Z toho $\varphi_e(H(e))$ je pevný bod. Protože:
	\[ \varphi_{f \circ \varphi_e(H(e))} (z) \simeq \varphi_{\varphi_{H(e)}(H(e))} (z) \stackrel{\text{diagonála}}{\simeq} \varphi_{\varphi_e(H(e))} (z) \]
\end{proof}

\section{\texorpdfstring{Aritmetická hierarchie}{Aritmetická hierarchie}}
\vspace{5mm}
\large

Q: k čemu je dobrá aritmetická hierarchie? Kde se používá/aplikuje? Totiž všechno nad Halting problémem není efektivně vyčíslitelné.

\begin{note}
	Aritmetická hierarchie je v jistém smyslu efektivní verzi \href{https://en.wikipedia.org/wiki/Borel_hierarchy}{Borelovské hierarchie}. Vezmeme konečně mnoho intervalu, budeme střidat $\cup, \cap$.

	$\cup$ odpovídá $\exists$ a $\cap - \forall$.
\end{note}

\begin{note}
	Podobná konstrukce jako polynomiální hierarchie v teorii složitosti.
\end{note}

\begin{definition}[$\Sigma_n, \Pi_n$]
	$\Sigma_n$ resp $\Pi_n$ prefix je skupina (aritmetických kvantifikátoru). $\Sigma_n$ začíná $\exists$, $\Pi_n$ naopak $\forall$.

	Každá ze skupin je homogenní - několik kvantifikátoru stejného typu, např $\exists\exists\exists$ nebo $\forall\forall$.
\end{definition}

\begin{definition}[Redukovaný prefix]
	Každá ze skupin obsahuje pouze jeden kvantifikátor.
\end{definition}

\begin{example}
	$\exists\exists\forall\exists\exists$ je $\Sigma_3$.

	$\exists\forall\exists$ je redukovaný $\Sigma_3$.
\end{example}

\begin{note}
	Aritmetická hierarchie se označuje $\Sigma_n^0, \Pi_n^0$ protože kvantifikace je přes $\N$ (aritmetická).
	Dolní index označuje počet střídavých kvantifikátoru.

	$\Sigma_n^1, \Pi_n^1$ by byla kvantifikace navíc přes funkce $f:\N \to \N$.
\end{note}

\begin{definition}
	Predikát (resp množina) je ve třídě $\Sigma_n (\Pi_n)$ jestliže je vyjadřítelný ve tvaru $\Sigma_n (\Pi_n)$ prefix na rekurzivní základ (ORP).

	Podobně pro relativní: \\
	$ \Sigma_n^{0, A}$, predikáty jsou $A$-ORP.
\end{definition}

\begin{observation}
	$\Sigma_0^0 = \Pi_0^0$ jsou právě rekurzivní predikáty.

Q: rovnost plyne z toho, že můžeme prohodit kvantifikace?
Na vyšších úrovních neplatí protože...??
\end{observation}

\begin{definition}[Aritmetický predikát]
	Predikát je \emph{aritmetický} právě když ho lze vyjádřít pomoci logiky 1. řádu, kde atomické části jsou rekurzivní.
\end{definition}

\begin{observation}
	Predikát je aritmetický právě když patří do $\Sigma_n$ nebo $\Pi_n$.
\end{observation}
\begin{proof}
	"$\Leftarrow$" zřejmé.

	"$\Rightarrow$" Úpravou výrazu do \href{https://en.wikipedia.org/wiki/Prenex_normal_form}{prenexního normálního tvaru}.
	Z logiky, libovolnou formuli lze do tohoto tvaru převést.
\end{proof}

\begin{note}
	Pokračováním do rekurzívních ordinálu lze studovat hyperaritmetickou hierarchii. Dal analytická hierarchie.

	V rámci kurzu končíme u konečných, neboli $\omega$.
\end{note}

\begin{example}\label{tot_pi2}
	Množina Tot \cref{tot_mn} je v $\Pi_2^0$.
\end{example}
\begin{proof}
	\[ x \in Tot \iff \forall y \exists x: \varphi_{x, s}(y) \downarrow \]
	Máme 2 střídavé kvantifikátory, $\varphi_{x, s}(y) \downarrow$ je rekurzivní.
\end{proof}

\begin{theorem}[Omezené kvantifikátory]
	Omezené kvantifikátory nezvyšují složitost (lze prohodit doprava).
\end{theorem}
\begin{proof}
	Rozebereme 2 případy dle typu omezeného kvantifikátoru na začátku formule ($\forall_{x \leq t}, \exists_{x \leq t}$).

	BUNO máme formuli v PNF:
	\[ \forall_{x \leq t} \exists y (\ldots) \]
	vezmeme $w$ jako kodovaní $(t + 1)$-tice, pak formuli lze ekvivalentně upravit na následující tvar
	\[ \exists w \forall_{x \leq t}(\ldots (w)_{t + 1, x} \ldots)\]
	Protože existence svědka pro všechny $x \leq t$ je stejný jako říct, že existuje skupina $(t + 1)$ svědků.

	Pak existenční kvantifikátor, BUNO máme formuli:
	\[ \exists_{x \leq t} \forall y (\ldots) \]
	Použijeme negaci a předchozí případ:
	\[ \forall_{x \leq t} \exists y \neg(\ldots) \]
	\[ \exists w \forall_{x \leq t}\neg(\ldots (w)_{t + 1, x} \ldots)\]
	Teď odstraníme negaci:
	\[ \forall w \exists_{x \leq t} (\ldots)\]
	V libovolné formuli můžeme postupem popsaném nahoře posunout omezené kvantifikátory doprava.
	Pak dle věty o omezené kvantifikace \cref{omez_kvant} ORP a omezený kvantifikátor jsou dohromady ORP.
	Omezený kvantifikátor lze nahradit konečnou disjunkce/konjunkce.

\end{proof}

\begin{theorem}[Redukovaný prefix]
	Libovolnou formuli lze převést do redukovaného prefixu.
\end{theorem}
\begin{proof}
	Znovu 2 případy dle typu kvantifikátoru.

	Podobně jako ve větě o neomezené kvantifikace \cref{neomez_kvant} nahradíme $n$ kvantifikátoru jediným kvantifikátorem $n$-tice.
	\[ \exists x \exists y \to \exists w ((w)_{2, 1} \ldots  (w)_{2, 2}) \]
	Analogicky pro $\forall$.
\end{proof}

\begin{example}
	Množina Rec $= \{ x : W_x\ \text{je rekurzivní}\ \} \in \Sigma_3^0$.
\end{example}
\begin{proof}
	Dle Postové věty \cref{post}:
	\[ x \in Rek \iff \exists y(W_x \cup W_y = W \land W_x \cap W_y = \emptyset) \]
	Neboli $W_y = \overline{W_x}$, dohromady vyplní celý prostor, ale průnik je prázdný.

	Přepíšeme formuli:
	\[ \forall y(\forall z (z \in W_x \cup W_y) \land \forall z (z \notin W_x \cap W_y)) \]
	Po krocích:
	\[ \forall y(\forall z \exists s(z \in W_{x, s} \cup W_{y, s}) \land \forall z \forall s(z \notin W_{x, s} \cap W_{y, s})) \]
	Pak šikovně vytáhneme jeden všeobecný kvantifikátor z levé části a 2 všeobecné z pravé části.
	V posledním kroku vytáhneme existenční kvantifikátor z levé části. Čímž dostaneme
	\[ \exists \forall \exists (\ldots) \in \Sigma_3^0 \]
\end{proof}

\begin{theorem}[Základní vlastnosti hierarchie]\label{aritm_hier_prop}
	\begin{enumerate}
		\item $A \in \Sigma_n \iff \overline{A} \in \Pi_n$
		\item $B \in \Sigma_n (\Pi_n) \Rightarrow \forall m > n: B \in \Sigma_m \cup \Pi_m$.
		%\item $\forall m \leq n: \Sigma_n \cup \Pi_n \subseteq \Sigma_m \cap \Pi_m$
		\item $A \leq_m B \land B \in \Sigma_n(\Pi_n) \Rightarrow A \in \Sigma_n(\Pi_n)$
	\end{enumerate}
\end{theorem}
\begin{proof}
	\begin{enumerate}
		\item Plyne z De Morgan pravidla. Negace mění kvantifikátor na opačný.
		\item Přidáme redundantní kvantifikátory přes fiktivní proměnné.

			Pokud jdeme směrem $\Sigma_n \to \Sigma_{n + 1}$, tak přidáme kvantifikátor na konec prefixe.
			Opačně $\Sigma_n \to \Pi_{n + 1}$, přidáme kvantifikátor na začátek prefixe.
		\item dle definice $\leq_m \exists f \in ORF$:
			\[ x \in A \iff f(x) \in B \]
			$f(x)$ můžeme jednodušé kvantifikovat.
	\end{enumerate}
\end{proof}

\subsection{Numerace}

\begin{theorem}
	Třída $\Sigma_0^0 = \Pi_0^0$ nemá univerzální ORP (rekurzivní numerace).
\end{theorem}
\begin{proof}
	Pomoci Cantorové diagonální metody.
	Nechť $R(e, x)$ je ORP.

	Pak musí platit:
	\[ \neg R(e, e) = R(a_0, e) \]
	položme $a_0 = e$ a dostáváme spor.
\end{proof}

\begin{note}
	Univerzální ČRF, neboli univerzální r.s. predikát \cref{univ_predic} je univerzální $\Sigma_1^0$ 2 proměnných pro třídu $\Sigma_1^0$ 1 proměnné.
\end{note}

\begin{theorem}[O numeraci, univerzálním predikátu]\label{aritm_numer}
	Pro $(n \geq 1)$ třída $\Sigma_n(\Pi_n)$ má univerzální $\Sigma_n(\Pi_n)$ predikát.

	Tedy máme $\Sigma_n(\Pi_n)$-indexu.
\end{theorem}
\begin{proof}
	Pro $\Sigma_n$, $\Pi_n$ analogicky.

	Nechť máme $\Sigma_n$ predikát.
	Nechť $n$ liché. Pak je tvaru
	\[ \exists \forall \ldots \exists Q(\ldots) \]
	Ořízneme poslední existenční kvantifikátor a predikát, dle věty o univerzálním r.s. predikátu \cref{univ_predic}:
	\[ \exists y_n Q(\ldots, y_n) = \exists y_n T_n(e, \ldots, y_n) \]
	Tím dostaneme vyjádření přes univerzální predikát:
	\[ \exists y_1, \forall y_2, \ldots \exists y_n T_n(e, y_1, y_2, \ldots, y_n) \]

	Pokud $n$ je sudé, tak máme predikát:
	\[ \exists \forall \ldots \forall Q(\ldots) \]
	Znovu použijeme negaci na
	\[ \forall Q(\ldots) = \exists \neg Q(\ldots) \]
	Což je r.s., proto se rovná univerzálnímu predikátu:
	\[ \exists \neg Q(\ldots) = \exists T_n(e, \ldots) \]
	zpět negace:
	\[ \exists T_n(e, \ldots) = \forall y_n \neg T_n(e, \ldots) \]
	Dohromady:
	\[ \exists y_1, \ldots, \forall y_n \neg T_n(e, y_1, \ldots, y_n) \]
\end{proof}

\begin{note}
	Ve třídě složitosti nemáme univerzální polynom, proto $P \neq NP$ problém.
\end{note}

\begin{consequence}
	Pro $(n \geq 1) \Sigma_n^0 - \Pi_n^0 \neq \emptyset$.
\end{consequence}
\begin{proof}
	Pro $n = 1$ máme $K \in \Sigma_1^0 - \Pi_1^0$.

	Pro ostatní $n$ stejný důkaz.
	Nechť $U(e, x)$ je univerzální predikát pro $\Sigma_n^0$. Kdyby $U(e, e) \in \Pi_n^0 \Rightarrow \neg U(e, e) \in Sigma_n^0$.
	Z existenci univerzálního $\neg U(e, e)$ má index $i$.
	Dosadíme index, dostaneme spor
	\[ U(i, i) = \neg U(i, i) \]
	Tedy $U(i, i) \notin \Pi_n^0$.
\end{proof}

\begin{definition}[$\Delta_n$]
	\[ \Delta_n = \Sigma_n \cap \Pi_n \]
\end{definition}

\begin{definition}[$\Sigma_n^0$-úplnost]
	$B$ je $\Sigma_n^0$-úplná právě když $B \in \Sigma_n^0$ a
	\[ \forall A \in \Sigma_n^0: A \leq_1 B \]
\end{definition}

\begin{theorem}[O aritmetické hierarchii]
	\begin{enumerate}
		\item $\emptyset^{(n)}$ je $\Sigma_n^0$-úplná pro $(n \geq 1)$.
		\item $A$ je r.s. v $\emptyset^{(n)} \iff A \in \Sigma_{n + 1}^0$.
		\item $A \leq_T \emptyset^{(n)} \iff A \in \Delta_{n + 1}$.
	\end{enumerate}

	Tato věta propojuje skok \cref{jump} s aritmetickou hierarchii a vyjádřitelnosti v PA.
\end{theorem}
\begin{proof}
	Indukci, pro $n = 0$ platí, protože
	\[ A\ \text{je r.s.} \iff A \in \Sigma_1^0 \]
	\begin{enumerate}
		\item z vlastnosti operace skoku \cref{jump} je $\emptyset^{(n + 1)}$ r.s. v $\emptyset^{(n)}$.
			Podle 2) $\emptyset^{(n + 1)} \in \Sigma_{n + 1}^0$.
			Pak $\emptyset^{(n + 1)}$ je $\Sigma_{n + 1}^0$-úplná.
		\item "$\Leftarrow$" Jelikož
			\[ A \in \Sigma_{n + 1}^0 \]
			$A$ Lze vyjádřit jako
			\[ \exists \forall \ldots Q(\ldots) \]
			Ořízneme od prvního kvantifikátoru
			\[ \forall \ldots Q(\ldots) \in \Pi_n^0 \]
			Použijeme trik s negaci jako ve větě o numeraci \cref{aritm_numer}.
			Čímž dostaneme predikát $P \in \Sigma_{n}^0$ který je dle i.p. $P \leq_1 \emptyset^{(n)}$.
			Tedy i $P \leq_T \emptyset^{(n)}$.
			Dáme zpět negace a dostaneme predikát tvaru
			\[ \exists (\emptyset^{(n)}\ \text{rekurzivní relace}) \]
			Z toho $A$ je r.s. v $\emptyset^{(n)}$.

		"$\Rightarrow$". Lze dokázat 2ma způsoby:

			a) Jelikož $A$ je r.s. v $\emptyset^{(n)}$. Tak
			\[ A = dom(\varphi) \]
			Kde $\varphi$ je $\emptyset^{(n)}$-ČRF.
			Dále
			\[ x \in A \iff \varphi(x) \downarrow \Rightarrow \Phi(\emptyset^{(n)})(x) \downarrow \]
			Kde $f = \Phi(\emptyset^{(n)})$ je %TODO
			Je to ekvivalentní
			\[ \exists \sigma \exists y(\Phi(\sigma)(x) \simeq y \land \sigma \prec \emptyset^{(n)}) \]
			Máme následující kvantifikátory:
			\[ \exists (\exists \land \sigma \prec \emptyset^{(n)}) \]
			Tvrdíme, že $\sigma \prec \emptyset^{(n)})$ je $\Sigma_n^0 \land \Pi_n^0$.
			Protože pro $j \leq |\sigma|$:
			\[ \sigma(j) = 1 \Rightarrow j \in \emptyset^{(n)} \]
			což dle i.p. je $\Sigma_n^0$.
			Opačně:
			\[ \sigma(j) = 0 \Rightarrow j \notin \emptyset^{(n)} \]
			což dle i.p. je $\Pi_n^0$.

			Dohromady:
			\[ \exists(\Sigma_n^0 \land \Pi_n^0)\]
			Vytáhneme existenční kvantifikátor
			\[\exists(\Pi_{n - 1}^0 \land \Pi_n^0) = \Sigma_{n + 1}^0 \]

			b) Jelikož $A$ je r.s. v $\emptyset^{(n)}$. Tak
			\[ A = dom(\varphi) \]
			Kde $\varphi$ je $\emptyset^{(n)}$-ČRF.

			Dle věty o limitní vyčíslitelnosti %todo link
			\[ \varphi(x) \simeq \lim_s F(x, s) \]
			kde $F$ je $\emptyset^{(n - 1)}$-ORF. Pak
			\[ x \in A \iff \varphi(x) \downarrow = \exists \lim_s F(x, s) \]
			Existenci limity lze zapsat:
			\[ \exists s_0 \forall t \geq s_0: F(x, t) = F(x, s_0) \]
			Protože jsme v diskretním prostoru, limita existuje když hodnota funkce se stabilizuje.
			Navíc $F(x, t) = F(x, s_0)$ je $\emptyset^{(n - 1)}$-ORF.
			Dle i.p. je $\Pi_n^0$ a $\Sigma_n^0$.
			Vezmeme jen $\Pi_n^0$ a dostaneme predikát:
			\[ \exists \forall(\Pi_n^0) \in \Sigma_{n + 1}^0 \]

		\item "$\Leftarrow$". Podle 2) $A$ je r.s. v $\emptyset^{(n)}$.
			Pak z vlastnosti hierarchie \cref{aritm_hier_prop}: $\overline{A} \in \Pi_n$.
			Takže $\overline{A}$ je r.s. v $\emptyset^{(n)}$.
			Dohromady dle Postové Věty \cref{rel_prop} A je rekurzivní v $\emptyset^{(n)}$.

			% TODO kde formálně indukční předpoklad?
			"$\Rightarrow$". Pokud $A \leq_T \emptyset^{(n)}$ tak podle indukčního předpokladu $A, \overline{A}$ je r.s. v $\emptyset^{(n)}$.
			Tedy $A \in \Delta_{n + 1}$.
	\end{enumerate}
\end{proof}

\begin{note}
	Předchozí věta souvisí s elementární aritmetikou, protože dle RDPM věty \cref{rdpm}:
	\[ \Sigma_1^0 \models_{\N} \Sigma_1-\text{formule} \]
	Což je taky rovnost dvou polynomů.
\end{note}

Z 1) bodů předchozí věty:
\begin{properties}
	\begin{enumerate}
		\item $K, \emptyset^{\prime}$ jsou 1-úplné neboli $\Sigma_1^0$-úplné.
		\item $\overline{K}, \overline{\emptyset^{\prime}}$ jsou 1-úplné neboli $\Pi_1^0$-úplné.
	\end{enumerate}
\end{properties}

\begin{theorem}[Tot úplnost]
	Množina Tot \cref{tot_mn} je $\Pi_2^0$-úplná.
\end{theorem}
\begin{proof}
	$Tot \in \Pi_2^0$, viz \cref{tot_pi2}.

	Nechť $B \in \Pi_2^0$ libovolná.
	Pak
	\[ x \in B \iff \forall y \exists s: Q(x, y, s) \]
	Kde $Q$ je rekurzivní predikát.
	Uděláme program:
	\[ \varphi_{\alpha(x)}(y) \simeq \mu_s Q(x, y, s) \]
	Pak
	\[ x \in B \Rightarrow \alpha(x) \in Tot \]
	Protože program najde $s$ pro všechna $y$.
	Opačně:
	\[ x \notin B \Rightarrow \alpha(x) \in Tot \]
	Protože $\exists y \forall s: \neg Q(x, y, s)$.
	Dokonce $\alpha$ lze udělat prostou.

	Z toho
	\[ B \leq_1 Tot \]
\end{proof}

\begin{definition}[Fin]
	\[ Fin = \{ x: |W_x| < \infty \} \]
\end{definition}

\begin{definition}[Inf]
	\[ Inf = \{ x: |W_x| = \infty \} \]
\end{definition}

\begin{theorem}[Fin úplnost]
	Množina Fin je $\Sigma_2^0$-úplná.
\end{theorem}
\begin{proof}
	\[ x \in Fin \iff \exists y \forall z, s (z \notin W_x \land z > y) \]
	Pokud je konečná, tak od určitého místa do ní nepadne žádný prvek.
	$y$ je buď maksimalní, nebo větší než max.

	Formule je $\Sigma_2^0$.

	Dokážeme přes komplement
	\[ \varphi_{\beta(x)}(y) \downarrow \iff \forall j \leq y (\varphi_x(j) \downarrow) \]
	Pak
	\[ x \in Tot \iff \beta(x) \in Inf \]
	Protože pokud není nekonečná, tak na nějakém vstupu nekonverguje a totiž není všude definováná.
	Taky opačně:
	\[ x \notin Tot \iff \beta(x) \notin Inf \]
	Alternativně:
	\[ \varphi_{\gamma(x)} (j) \downarrow \iff \exists j-\text{prvků}\ \in W_x \]
	Pak platí
	\[ x \in Tot \iff \gamma(x) \in Inf \]
	a i opačně.
	Dohromady:
	\[ Inf \equiv_1 Tot \]
\end{proof}

\begin{definition}[Rek]
	\[ Rek = \{ x: W_x\ \text{je rekurzivní}\ \} \]
\end{definition}

\begin{theorem}[Rek úplnost (BD)]
	Množina Rek je $\Sigma_3^0$-úplná.
\end{theorem}

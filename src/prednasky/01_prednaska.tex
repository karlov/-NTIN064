\section{\texorpdfstring{Zkratky}{Zkratky}}
\vspace{5mm}
\large

\begin{enumerate}
	\item ČRF - částečně rekurzivní funkce.
	\item ORF - obecně rekurzivní funkce.
	\item PRF - primitivně rekurzivní funkce.
	\item r.s. - rekurzivně spočetná.
	\item ZAS - zakladní aritmetika síla.
	\item PA - Peano Aritmetika.
	\item ORP - obecně rekurzivní predikát.
	\item PNF - prenexní normální tvar.
\end{enumerate}

\section{\texorpdfstring{Uvod}{Uvod}}
\vspace{5mm}
\large


\subsection{Historická vsuvka}

Hilbertův 10. problém, úplnost aritmetiky 1900.
G\"{o}del dokázal že nejde. V prvním větě použil \emph{Primitivně rekurzivní funkce}.

\begin{definition}
	Primitivně rekurzivní funkce - podmnožina efektivně vyčíslitelných funkci, jsou všude definované.
\end{definition}

Tyto ale nestačí pro hlavní problém dokazatelnosti.

Při dalším vývoje se vyvinul kalkulus tzv obecně rekurzivních funkci ORF a částečně rekurzivních funkci ČRF.

\begin{definition}
	Částečně rekurzivní funkce - efektivně vyčíslitelné funkce.
\end{definition}

\begin{definition}
	Obecné rekurzivní funkce - ČRF které jsou všude definované.
\end{definition}

\begin{note}[Church-Turing teze]
	Historický vývoj:
	\begin{itemize}
		\item A. Church vyvinul $\lambda$-konverze ($\lambda$-calculus) a dokázal, že neexistuje algoritmus tzv "rozhodovací".
		Lambda konverze jsou poměrně obtížné.
		\item Turing, nezávislé na Churchoví v roce 36 vyvinul Turing Machines a dokázal nevyčíslitelnost Halting problému.
	\end{itemize}

	Pak ostatní prohlásili Church-Turing teze, že všechno co je efektivně vyčíslitelné je Turingovský nebo $\lambda$-konverzi vyčíslitelné.
\end{note}

\begin{definition}
	$\lambda$-calculus:

	Nechť $C$ je množina konstant, nechť $V$ je (spočetná) množina proměnných.
	Množina tzv $\lambda$ terms $\Lambda$ je nejmenší množina tž:
	\begin{itemize}
		\item $C \subseteq \Lambda$.
		\item $V \subseteq \Lambda$
		\item nechť $t_1, t_2 \in \Lambda$ termy, pak aplikace $t_1\ t_2$ jako v Haskellu je taky term
		\item $t \in \Lambda, x \in V \Rightarrow \lambda x . t \in \Lambda$.
			V Haskellu:\\ $(\textbackslash x \to t)$

			Což je funkce s parametrem $x$ a vrací $t$.
	\end{itemize}
\end{definition}

Jako závěr, formální, efektivně dokazovací systém nemůže uplně popsat pravdu. Je mnohem složitější.

\begin{note}[System PRF(odbočka)]
	Funkcionální systém, postavený na axiomech:
	\begin{itemize}
		\item Základní funkce: 0, +1, id (resp vydělení i-té složky)
		\item 2 Odvozovací pravidla:\\
			1) substituce \\
			2) operátor primitivní rekurze.
			Jednoduše řečeno, výpočet v bode $(y + 1)$ uděláme rekurzivně z bodu "$y$".
	\end{itemize}

	Pak se vezme \emph{tranzitivní uzávěr} - všechno co jde odvodit ze základních funkci pomoci odvozovacích pravidel.
	Na rozdíl od ČRF nemáme \textbf{while}, čímž dostaneme jenom podmnožinu ČRF.

	Substituce:
	\[ S(f, g_1, \ldots, g_n) = f(g_1(y_1, \ldots, y_n), \ldots, g_n(y_1, \ldots, y_n)) \]

	Primitivní rekurze:
	%todo 2nd lecture od 09:00

\end{note}

\begin{note}[Kleeneho system ČRF]
	Pak přidáním operátoru $\mu$ a \textbf{while} k předchozímu systému dostaneme ČRF (znovu \emph{tranzitivní uzávěr}).
	Je komplikovaný, je lepší používat nějaké dokazování.
\end{note}

Q: je možné, že existuje mnohem komplikovanější systém než všechny které máme v Church-Turingové téze, který by byl silnější z pohledu vyčíslitelnosti?

\subsection{Terminologie}

Přibližně do roku 1990 převládala terminologie ORF, ČRF zavedená Kleene.
Pak byla snaha změnit na \textbf{computable functions} - efektivně vyčíslitelné.
\begin{definition}
	Množina je rekurzivní, neboli rozhodnutelná (decidable, computable) - efektivně rozhodnutelná.
	Jednoduše řečeno, máme program, který na každém vstupu se vždy zastaví a rozhodne ANO nebo NE (jestli slovo patří do ni).
\end{definition}

\begin{definition}
	Množina je rekurzivní spočetná (částečně rozhodnutelná), nebo computably enumerable.
	Formálně je definičním oborem nějakého programu (tzn částečně rekurzivní funkce, TS etc.).
\end{definition}

\begin{note}
	Na rozdíl od kurzu ZSV, kde jsme definovali funkce $f:\{0, 1\}^n \to \{0, 1\}^n$,
	budeme zkoumat funkce aritmetické.

	\[ f:\N \to \N, f:\N^k \to \N \]
	Nemusí být všude definované.

	Přístupy jsou ekvivalentní, protože můžeme očíslovat slova.
\end{note}

\begin{note}
	Formálně nemáme klasické k-tice jako vektory, ale kodujeme všechno do přirozených čísel.
	Je znamá jako \href{https://en.wikipedia.org/wiki/Pairing_function}{Cantorová metoda parovaní}.
\end{note}

\begin{notation}
	Program \textbf{konverguje} - znamená že se zastaví za konečný počet kroků.
\end{notation}

\begin{reminder}
	Rekurzivita, rekurzivní spočetnost se zachovává na $\cup, \cap$.

	Rekurzivita je taky zachovaná při $\neg$ (doplněk).
	Rekurzivní spočetnost nikoliv.
\end{reminder}

\begin{theorem}[Postova]\label{post}
	$L$ je rozhodnutelný $ \iff L \land \bar{L} $ jsou c.r.
\end{theorem}
\begin{proof}
$ \Rightarrow $. Z TS pro $L$ sestavíme pro doplněk znegovaním všech odpovědi. \\
$ \Leftarrow $. Nechť $L(M) = L \land L(B) = \bar{L} $, pak sestavíme TS pro rozhodnutí $L$.

\begin{alltt}
1. Pust B, M paralelně
2. if(Acc(M, x))
3. \tab accept
4. if(Acc(B, x))
5. reject
\end{alltt}

Pokud se aspoň 1 zacykli - reject. Paralelní spuštění lze implementovat pomoci 2 pasek, případně je slepit do 1.
\end{proof}

\begin{definition}
	G\"{o}delové číslo - index programu.
	Nechť $\varphi$ je ČRF, $P_e$ je program který ji vyčísluje.
	Pak index funkce $\varphi$ je $e$.
\end{definition}

\begin{note}
	Každá ČRF má nekonečně mnoho programu, takže i nekonečně mnoho indexu.
	Očíslovaní programu generuje očíslovaní funkci.

	V jistém smyslu, nezáleží na konkretním očíslovaní pokud je efektivní (nemáme čas toto dokazovat).
	2 různé indexace jsou \emph{efektivně ekvivalentní}.
\end{note}

Q: pokud zafixujeme "jazyk programování" má program jednoznačné očíslovaní?
Q: jaký z jazyků programování" je nejbližší k ČRF? Asi $\lambda$-calculus.

\begin{reminder}
	Univerzální TS - dostane program $M$ a data $x$, simuluje výpočet $M(x)$.

	Pro nás to bude \emph{univerzální ČRF}.
\end{reminder}

\begin{definition}
	Univerzální ČRF je
	\[ \Psi_n(e, x_1, x_2, \ldots, x_n) \]
	kde $e$ je index programu, $x_i$ jsou data.

	Občas se značí
	\[ \varphi_e^n(x_1, \ldots, x_n) \simeq {e}(x_1, \ldots, x_n) \]
\end{definition}

\begin{notation}
	\[ \varphi_e^n(x_1, \ldots, x_n) \simeq U(\mu_y T_n(x_1, \ldots, x_n, y)) \]

	\begin{itemize}
		\item Kde $T_n$ je primitivně rekurzivní predikát, který říká "za $n$ kroků".
		\item $U$ je primitivně rekurzivní funkce 1 proměnné.
			Který "vydělí" výsledek z mezivýsledků (jelikož máme všechno zakodované jako přirozená čísla).
		\item $\mu_y$ říká "nejmenší $y$".
	\end{itemize}
\end{notation}

\begin{theorem}[s-m-n (BD)]\label{s_m_n}
	\[ \varphi_e^{m + n}(x_1, \ldots, x_n, y_1, \ldots, y_m) \simeq \varphi_{s_n^m(e, y_1, \ldots, y_m)}^n(x_1, \ldots, x_n) \]
	V $\Psi$ notace
	\[ \Psi_{n + m}(e, \bar{x}, \bar{y}) \simeq \Psi_m(s_n^m(e, \bar{x}, \bar{y})) \]
	kde $\bar{x}, \bar{y}$ jsou vektory pro kratší zápis.

	Funkce $s_n^m: e, x_1, \ldots, x_n$ vyrobí nový program.
	Ten čeká na vstup $y_1, \ldots, y_m$, k tomu přidá zahardkodované data $x_1, \ldots, x_n$ a spustí na to $e$.
	Je to pouze syntaktická manipulace dat.
\end{theorem}

\begin{notation}
	$dom(\varphi)$ - definiční obor.
\end{notation}
\begin{notation}
	$range(\varphi)$ - obor hodnot.
\end{notation}

\begin{definition}
	$e$-ta rekurzivně spočetná množina.
	\[ W_e = dom(\varphi_e) = \{ x: \varphi_e(x) \downarrow \} = \{ \Psi_1(e, x) \downarrow \} \]
\end{definition}

\begin{note}
	Rekurzivní spočetné funkce se definuji jako obor hodnot ČRF.

	$M$ je rekurzivní množina $\iff$ je oborem hodnot \textbf{rostoucí} ČRF.

	$M$ je rekurzivně spočetná množina $\iff$ je oborem hodnot \textbf{prosté} ČRF.

	Rozdíl v definici souvisí s Halting problémem.
\end{note}

\begin{definition}
	$A \leq_1 B \iff \exists$ ORF $f$ (všude definovaná, efektivně vyčíslitelná):
	\[ x \in A \iff f(x) \in B \]
\end{definition}

\begin{notation}
	\[ K = \{ x: s \in W_x \} = \{ s: \varphi_x(x) \downarrow \} \]
	Taky
	\[ K_0 = \{ \langle x, y \rangle: \varphi_x(y) \downarrow \} \]
	Značení z ZSV
	\[ DIAG = \{ \langle M \rangle: M \in L(M) \} = \{ \langle M \rangle: M(\langle M \rangle) \} \]
\end{notation}

\begin{note}
	DIAG je rekurzivně spočetný (částečně rozhodnutelný) ale není rekurzivní (rozhodnutelný).

	Důkaz pomoci Cantorové diagonální metody.
\end{note}
\begin{proof}
	\[ \overline{K} = \{ x: x \notin W_x \} \]
	$W_x$ ale jsou všechny rekurzivně spočetné.
	Z toho
	\[ \forall W_x: \overline{K} \neq W_x \]
	Takže $\overline{K}$ není částečně rekurzivní.
	Dle Postové věty \cref{post} K není rozhodnutelná.
\end{proof}

\begin{theorem}[K 1-complete]
	K (taky $K_0$) je 1-úplná.
\end{theorem}
\begin{proof}
	Zavedeme ČRF
	\[ \alpha(x, y, w) \downarrow \iff \varphi_x(y) \downarrow\]
	kde $w$ je fiktivní proměnná.
	Je to ekvivalentní
	\[ \Psi_1(e, x, y, w) \]
	použijeme \cref{s_m_n}
	\[ \Psi_1(e, x, y, w) \simeq \Psi_1(s_2^1(e, x, y), w) \simeq \varphi_{s_2^1(e, x, y)}(w) \]
	dosadíme za $w = s_2^1(e, x, y)$.

	Pomoci $w$ se dostáváme na diagonálu. Pak
	\[ x \in W_y \iff s_2^1(e, x, y) \in K \]
\end{proof}

\begin{theorem}
	$\Psi_n$ nemá obecně rekurzivní rozšíření.
	Jinými slovy neexistuje ORF $h$ rozšíření $\Psi$ tž
	$\Psi_n(x) = h(x)$ pro $x \in dom(\Psi_n)$ a $h$ je definovaná pro vstupy mimo $dom(\Psi_n)$.

	Dokonce, pokud $\alpha$ částečně rekurzivně rozšiřuje $\Psi_n$, tak najdeme vstup na kterém diverguje
	\[ \exists x_1: \alpha(x_1, x_1) \uparrow \]
\end{theorem}
\begin{proof}
	Použijeme Cantorovou diagonální metodu.
	Definujme pomocnou ČRF:
	\[ \beta(x) \simeq 1 \dotminus \alpha(x, x) \]
	Kde $\dotminus$ je dodefinovaná operace odečítaní pro přirozená čísla.
	Např $1 \dotminus 100 = 0$.
	Jelikož je ČRF $\Rightarrow$ má index $e_{\beta}$, neboli
	\[ \beta(e_{\beta}) \simeq \Psi_1(e_{\beta}, e_{\beta}) \simeq 1 \dotminus \alpha(e_{\beta}, e_{\beta}) \]
	Nechť sporem $\alpha(e_{\beta}, e_{\beta}) \downarrow$, pak
	\[ \Psi_n(e_{\beta}, e_{\beta}) \downarrow \]
	Protože $\alpha$ je rozšíření
	\[ \Psi_n(e_{\beta}, e_{\beta}) = \alpha(e_{\beta}, e_{\beta})\]
	což je spor protože
	\[ 1 \dotminus \alpha(e_{\beta}, e_{\beta}) = \Psi_n(e_{\beta}, e_{\beta}) = \alpha(e_{\beta}, e_{\beta})\]
\end{proof}

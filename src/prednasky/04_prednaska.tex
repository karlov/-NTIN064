\section{\texorpdfstring{Produktivní množiny}{Produktivní množiny}}
\vspace{5mm}
\large

\begin{definition}[Produktivní množina]
	B je \emph{produktivní} pokud
	\[ \exists \varphi \in ČRF: W_x \subseteq B \Rightarrow (\varphi(x) \downarrow) \land \varphi(x) \in B \setminus W_x \]

	Jinými slovy: non-rekurzivní spočetnost. Pokud máme uvnitř množinu $W_x$ tak se nemůže rovnat B.
	Taky máme stroječek který najde $\varphi(x)$ který leží mimo danou $W_x$.
\end{definition}

\begin{definition}[Kreativní množina]
	Množina A je \emph{kreativní} pokud A je rekurzivně spočetná a $\overline{A}$ je produktivní.
\end{definition}

\begin{example}
	$\overline{K}$ je produktivní funkce je $id$, $K$ je kreativní.

	Protože
	\[ W_x \subseteq \overline{K} \Rightarrow x \in (\overline{K} - W_x) \]
\end{example}

\begin{theorem}[Modifikace K]
	Modifikace předchozího příkladu:\\
	Nechť máme $f \in ORF$ prostá, pak uvažme množinu:
	\[ A = \{ f(x)| \ f(x) \in W_x \} \]
	A je kreativní, $\overline{A}$ produktivní s $f$.
\end{theorem}
\begin{proof}
	Nechť $W_x \subseteq \overline{A}$, kdyby $f(x) \in W_x$ tak
	\[ f(x) \in \overline{A} \]
	ale dle definice A
	\[ f(x) \in A \]
	Tedy
	\[ f(x) \notin W_x \]
	a jelikož je \textbf{prostá} tak
	\[ f(x) \in \overline{A} - W_x \]
\end{proof}

\begin{theorem}[Produktivní funkce ORF]
	Každá produktivní množina má ORF produktivní funkce.
\end{theorem}
\begin{proof}
	Jednoduše dodefinovat ČRF na ORF nejde.

	Chceme najít ORF $h$:
	\[ W_{h(y)} = \twopartdef { W_y } { \varphi(h(y)) \downarrow } { \emptyset } { \varphi(h(y)) \uparrow }, kde\ \varphi \in ČRF\ prod. \]

	Formálně:
	\[ \varphi \circ h \in ORF \]
	Kdyby $\varphi(h(y)) \uparrow$ tak
	\[ \Rightarrow W_{h(y)} = \emptyset \subseteq B \Rightarrow \varphi(h(y)) \downarrow spor \]
	Dal
	\[ \forall y: W_{h(y)} = W_y \]
	Taky
	\[ W_y \subseteq B \Rightarrow W_{h(y)} \subseteq B \]
	Z toho
	\[ \varphi(h(y)) \in B - W_{h(y)} = B - W_y \]

	Hledaná funkce je $\varphi \circ h$.

	Ziskáme funkci $h$ pomoci věty o rekurzi \cref{rek_1}.
	Vezmeme pomocnou $f \in ORF$:
	\[ W_{f(x, y)} = \twopartdef { W_y } { \varphi(x) \downarrow } { \emptyset } { \varphi(x) \uparrow } \]
	f pomoci s-m-n \cref{s_m_n}
	\[ f \simeq \alpha(x, y, w) \downarrow \iff w \in W_y \land \varphi(x) \downarrow \]
	Taky
	\[ \alpha(x, y, w) \simeq \varphi_{f(x, y)} (w) \]
	kde
	\[ f(x, y) = s_2(a, x, y) \]
\end{proof}

\begin{theorem}[Produktivní funkce ORF prostá(BD)]
	Každá produktivní množina má dokonce prostou ORF produktivní funkce.

	Dokonce rekurzivní permutace.
\end{theorem}

\begin{theorem}[Nekonečná množina]\label{product_inf}
	Každá produktivní množina obsahuje nekonečnou r.s. podmnožinu.
\end{theorem}
\begin{proof}
	Máme $B$ a $f \in ORF$ produktivní.

	Vezmeme takové $z_0$:
	\[ W_{z_0} = \emptyset \]
	Množinu vytváříme iterativně, vždy na jeden z bodů co máme aplikujeme $f$ a vezmeme sjednocení.

	Formálně:
	\[ W_{g(x)} = W_x \cup \{ f(x) \} \]
	rekurze
	\begin{align}
		h(0) = z_0\\
		h(y + 1) = g(h(y))
	\end{align}
	Pak
	\[ W_{h(y)} = \{ f(z_0), \ldots, f(h(y) - 1) \} \]
	což je $y$ bodů z B.
\end{proof}

\begin{note}
	Imunní a produktivní množiny jsou disjunktní pojmy.
\end{note}

\begin{amendment}
	Jak dlouho lze pokračovat v konstrukci množiny popsané ve Větě o nekonečné množině \cref{product_inf}?

	Odpověď: pokud to bude efektivní proces neboli aby množiny byly r.s.

	Můžeme iterovat $\omega, 2 \omega \ldots$ podél tzv rekurzivních ordinálů (viz ordinální číslo v teorii množin).
\end{amendment}

\begin{lemma}\label{prod_prev}
	A produktivní a $A \leq_m B$.

	Neboli produktivita se zachovává směrem vzhůru při $\leq_m$.
\end{lemma}
\begin{proof}
	Máme ORF funkce $g$ z převoditelnosti.
	Pak nechť $W_x \subseteq B$, najdeme její preimage v A
	\[ P = g^{-1}(W_x) \subseteq A \]
	Z toho že A je kreativní, pomoci kreativní funkce $f$ najdeme bod $f(y) \notin P$.
	Zobrazíme pomoci $g(f(y))$, tím dostaneme bod $\in B - W_x$.

	Formálně:
	\[ W_{h(x)} = g^{-1}(W_x) = \{ y |\ g(y) \in W_x \} \]
	Pak
	\[ W_x \subseteq B \Rightarrow W_{h(x)} \subseteq A \]
	Poslední krok
	\[ g \circ f \circ g^{-1} (x) \in B - W_x \]

\end{proof}

\begin{theorem}[Ekvivalence Kreativní]
	Nechť $M$ množina.
	Následující tvrzení jsou ekvivalentní:
	\begin{enumerate}[label=(\alph*)]
	    	\item $M$ je kreativní $\iff \overline{M}$ produktivní.
		\item $M$ je 1-úplná $\iff \overline{K} \leq_1 \overline{M}$
		\item $M$ je m-úplná $\iff \overline{K} \leq_m \overline{M}$
	\end{enumerate}
	Každý z pojmu zahrnuje rekurzivní spočetnost.

	Ekvivalence mezi totálně různými pojmy. 1-úplnost jako u NP znamená, že je to nejtěžší ze všech takových množin.
\end{theorem}
\begin{proof}
	$(b) \Rightarrow (c)$ z vlastnosti 1 a $m$ převoditelnosti.

	$(c) \Rightarrow (a)$ \\
	Z vlastnosti převoditelnosti
	\[ K \leq_m M \iff \overline{K} \leq_m \overline{M} \]
	pak použijeme lemma \cref{prod_prev}.
	Víme že $\overline{K}$ je produktivní, takže i $\overline{M}$.
	Pak dle definice, $M$ je kreativní.

	$(a) \Rightarrow (b)$ ($\overline{M}$ produktivní $\Rightarrow \overline{K} \leq_1 \overline{M}$)

	Cil
	\[ W_{h(y)} = \twopartdef { \{ f \circ h(y) \} } { y \in K } { \emptyset } { y \notin K } \]
	kde $f$ je ORF prostá, produktivní pro $\overline{M}$.

	Konstrukce funkce $h$
	\[ W_{g(x, y)} = \twopartdef { \{ f(x) \} } { y \in K } { \emptyset } { y \notin K } \]
	$g$ dostaneme pomoci s-m-n věty \cref{s_m_n}:
	\[ \alpha(x, y, w) \simeq \varphi_{g(x, y)} (w) \downarrow \iff y \in K \land w = f(x) \]
	Strojil skripta chyba, rovnice č. 57.

	Pak použijeme větu o rekurzi
	\[ W_{h(y)} = W_{g(h(y), y)} \]

	Z toho platí
	\begin{gather*}
		y \notin K \Rightarrow W_{h(y)} = \emptyset \subseteq \overline{M} \Rightarrow f \circ h(y) \in \overline{M} \\
		y \in K \Rightarrow W_{h(y)} = \{ f \circ h(y) \}
	\end{gather*}
	kdyby $f \circ h(y) \in \overline{M}$ tak
	\[ W_{h(y)} \subseteq \overline{M} f \circ h(y) \in \overline{M} - W_{h(y)} \]
	což je spor.

	Neboli
	\[ f \circ h(y) \in M \Rightarrow \overline{K} \leq_1 \overline{M} \]
\end{proof}

\begin{consequence}
	$\overline{K}$ je nejjednodušší produktivní množinou při $\leq_1$ nebo $\leq_m$.
	Protože všechny produktivní množiny jsou
	\[ \{ B |\ \overline{K} \leq_m B \} \]
\end{consequence}

\begin{definition}
	B je úplně produktivní když existuje ORF $f$ tž:
	\[ f(x) \in B - W_x \lor f(x) \in W_x - B \]
\end{definition}

\begin{example}
	$\overline{K}$ je úplně produktivní dle definice K.
	\[ x \in \overline{K} - W_x \lor x \in W_x - \overline{K}\]
	neboli funkce je $id$.
\end{example}

\begin{theorem}[Uplna produktivita]
	B je úplně produktivní $\iff$ B je produktivní.
\end{theorem}
\begin{proof}
	$\Rightarrow$ triviálně z definice.

	$\Leftarrow$ lze dokázat 2ma způsoby.
	První je inspekci minulého důkazu.
	Uděláme
	\begin{enumerate}
		\item $g^{-1}(W_x)$
		\item $f(\ldots)$
		\item $g \circ f \circ g^{-1}$.
	\end{enumerate}
	Jen se musí ověřit o 1 disjunkci víc.

	Druhý pomoci věty o rekurzi:
	\[ W_{h(y)} = \twopartdef { \{ f \circ h(x) \} } { f \circ h(x) \in W_y } { \emptyset } { f \circ h(x) \notin W_y } \]
	f je ORF produktivní funkce, $h$ dostaneme pomoci věty o rekurzi a s-m-n věty. Pak
	\[ f \circ h(x) \notin W_y \Rightarrow W_{h(y)} = \emptyset \Rightarrow f \circ h(x) \in B \Rightarrow f \circ h(x) \in B - W_y \]
	\[ f \circ h(x) \in W_y \Rightarrow W_{h(y)} = \{ f \circ h(x) \} \]
	Kdyby $f \circ h(x) \in B$ tak
	\[ \Rightarrow W_{h(y)} \in B \Rightarrow f \circ h(x) \in B - W_{h(y)} \]
	Z toho
	\[ f \circ h(x) \in W_y - B \]
\end{proof}

\begin{definition}[Totální množina]\label{tot_mn}
	\[ Tot = \{ x |\ \varphi_x\ \text{totální} \} = \{ x |\ \exists y \varphi_x(y) \downarrow \} \]
\end{definition}
\begin{lemma}[Totalni je produktivni]
	Totální množina je produktivní.
\end{lemma}
\begin{proof}
	Pomoci $m$-převodu na $\overline{K}$.

	\[ \varphi_{h(x)} (y) \downarrow \iff x \notin K_j \]
	Kde $x \notin K_j$ znamená, že $x \notin K$ za $j$ kroků.

	\[ x \notin K \iff h(x) \in Tot \]
	Pokud $x$ není v $K$, tak tam nebude za žádný počet kroků. Pak i $h(x)$ je všude definovaná.

	Jinak
	\[ x \in K \Rightarrow dom(\varphi_{h(x)}) < \infty \]
	Definiční obor je konečný a rovna se nějakému $\{ 0, ..., j_0 \}$, což je počet kroků za který $x$ vstoupí do $K$.

	Problém ale je, že dostáváme nový program, ale ne zaručeně novou funkci.

	Dokážeme silnější tvrzení a konkretně vytvoříme novou funkci.
	Máme
	\[ W_y \in Tot \]
	uděláme novou $F \in ORF$ která roste rychleji než $\varphi_a: \forall a \in W_y$.
	Jinými slovy
	\[ \forall a \in W_y \exists z_0 \forall z \geq z_0: F(x) \geq \varphi_a(z) \]
	$F$ majorizuje $\varphi_a: \forall a \in W_y$.

	BUNO: $W_y$ je nekonečná, jinak přidáme nekonečně indexů prázdného programu.
	Kvůli enumeratoru, můžeme $W_y$ efektivně generovat, neboli vypisovat
	\[ a_0, a_1, \ldots \]

	Pak
	\[ F(x) = \max_{j \in \{ 1, \ldots, x \} } (\varphi_{a_j}(x)) + 1 \]
\end{proof}

\begin{consequence}
	Z věty plyne omezení logiky.

	Vezmeme třeba Peano aritmetiku (PA).
	Můžeme efektivně generovat sentence které PA dokazuje, tudíž lze efektivně generovat ty $a: \varphi_a$ totální.

	Pak můžeme dle předchozí věty můžeme najít $F$ která roste rychleji, než cokoliv co PA dokazuje.

	Libovolná efektivně zadaná teorie má jen r.s. množinu dokazatelných sentenci.
	Pokud z nich vybereme ty, co dokazuji o nějakém programu že je všude definovaný, tak vyrobíme sentenci na kterou daná teorie nestačí.
\end{consequence}

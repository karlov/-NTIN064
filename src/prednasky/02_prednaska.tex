\section{\texorpdfstring{Rekurzivně spočetné množiny a predikáty}{Rekurzivně spočetné množiny a predikáty}}
\vspace{5mm}
\large

\begin{note}
	R.s. množiny a predikáty je jedno a totéž protože obor pravdivostí predikátu je množina a naležení do množiny je predikát.
\end{note}

\begin{lemma}
	Pokud $Q$ je rekurzivní $\Rightarrow \exists y Q(...)$ je rekurzivně spočetný.
\end{lemma}
\begin{proof}
	Uvažme charakteristickou funkci $C_Q$ predikátu $Q$.
	Je všude definovaná, čili je ORF.

	Pak následující je ČRF:
	\[ \mu_y Q(...) \simeq \mu_y (C_Q(...) = 1) \]
\end{proof}

\begin{theorem}[Univerzální Kleeneho r.s. predikát]\label{univ_predic}
	Každý rekurzivně spočetní predikát je tvaru:
	\[ \exists y Q(...) \]

	Pak r.s. množiny jsou definiční obory ČRF.

	Dokonce máme univerzální rekurzivně spočetný predikát
	\[ \exists y T_n(e, x_1, \ldots, x_n, y) \]
\end{theorem}

\begin{consequence}
	Lze definovat index rekurzivně spočetných predikátů.
\end{consequence}

\begin{note}
	s-m-n věta \cref{s_m_n} platí i pro predikáty $T_n$.
\end{note}

\begin{theorem}
	Rekurzivní spočetnost je uzavřená na $\cup, \cap$.
	Dokonce efektivně z indexu.

	Máme ORF(dokonce PRF)
	\[ W_{\alpha(a, b)} = W_a \cap W_b \]
\end{theorem}
\begin{proof}
	Formálně pro $\cap$:
	\[ \exists s_1 T_1(a, x, s_1) \land \exists s_2 T_1(a, x, s_2) \iff \exists w(T_1(a, x, (w)_{2,1}) \land T_1(b, x, (w)_{2,2}) \]
	kde $w$ koduje dvojici $s_1, s_2$.
	\[ (w)_{2,1} \]
	říká, že $w$ je n-tice velikosti 2, vezmi 1. složku.

	\[ \exists s_1 T_1(a, z, s_1) \]
	je reprezentace množiny $W_a$ pomoci univerzálního predikátu.

	Dohromady máme rekurzivně spočetný predikát. Takže
	\[ \exists z T_3(e, a, b, x, z) \]
	Což je program, který pouští oba dva programy a čeká až se jeden z nich zastaví.
	Použijeme s-m-n \cref{s_m_n} pro predikáty
	\[ \exists z T_3(e, a, b, x, z) \iff \exists z T_1(s_2(e, a, b), x , z) \]
	Pak definujme
	\[ \alpha(a, b) := s_2(e, a, b) \]

	Analogicky pro $\cup$.
\end{proof}

Otázka: pokud všude použijeme omezené kvantifikátory s $y =$ velikost částic ve vesmíru, dostaneme upravenou logiku pro počítače?

\begin{definition}
	Omezený existenční kvantifikátor
	\[ \exists_{y < z} Q \]
	Jmenuje se konečná dizjunkce.
\end{definition}

\begin{definition}
	Omezený všeobecný kvantifikátor
	\[ \forall_{y < z} Q \]
	Jmenuje se konečná konjunkce.
\end{definition}

\begin{theorem}[Omezená kvantifikáce]\label{omez_kvant}
	Rekurzivní spočetnost je uzavřená na omezené kvantifikáce.

	Dokonce efektivně na indexech, ale toto dělat nebudeme.
\end{theorem}
\begin{proof}
	Pro existenční spustíme $z$ programů paralelně a čekáme až jeden přijme.

	Pro všeobecný spustíme paralelně a čekáme jestli všechny přijmou.
\end{proof}

\begin{theorem}[Neomezená kvantifikáce]\label{neomez_kvant}
	Rekurzivní spočetnost je uzavřená na existenční kvantifikáce.
\end{theorem}
\begin{proof}
	Analogicky jako důkaz pro $\cap$, nahradíme dva existenční kvantifikátory jediným s dvojici.
	\[ \exists y \exists s: T_n(...) \simeq \exists k = \langle y, s \rangle ... \]
\end{proof}

\begin{note}
	Pro všeobecnou již neplatí (ani pro částečně rozhodnutelné).
	Protipříkladem je $\overline{K}$ která není č.r.
	Již lze zapsat pomoci všeobecného kvantifikátory
	\[ x \in \overline{K} \iff \forall s \neg T_1(x, x, s) \]
	Kde $T_1$ je částečně rozhodnutelný predikát, negace taky.
\end{note}

\subsection{10 Hilbertův problém}

V moderní terminologií 10. Hilbertův problém zni: "zda existuje algoritmus, který by pro libovolný celočíselný polynom rozhodnul jestli existuje řešení v celých číslech.

Libovolný celočíselný polynom je ekvivalentní 2 polynomům v $\N$, řešení pak taky hledáme v $\N$.
Nejprve dáme záporné koeficienty na pravou stranu, pak aplikujeme Lagrangeovou větu o 4 $\square$.

\begin{theorem}[RDPM (BD)]\label{rdpm}
	Predikát Q je rekurzivně spočetný $\iff$ je tzv diofantický:
		\[ \exists x_1, \ldots, x_k \in \N: (p_1(x_1, \ldots, x_k, y_1, \ldots, y_n) = p_2(x_1, \ldots, x_k, y_1, \ldots, y_n)) \]
\end{theorem}

\begin{consequence}
	10. Hilbertův problém má negativní odpověď.

	Protože máme množiny které nejsou rozhodnutelné, třeba DIAG.
\end{consequence}

\begin{amendment}\label{rdpm_am}
	Jako byprodukt dostáváme ekvivalenci:
	% todo PRP = primitivně rekurzivní predikát??
	\[ \exists(PRP) \iff \exists(p_1(\ldots) = p_2(\ldots)) \]
	Bez existenčního kvantifikátoru vůbec není pravda.
	V aritmetice lze vytvořit i superexponenxielu $e^{n^n}$ atd, polynomy jsou ale omezené.
	Taky polynomy lze elementárně vyjádřit pomoci Robinsonové aritmetiky.
\end{amendment}

\subsection{Selektory}

Obecné, Selektor je definovaný pro "hezké relace", např $Q(x, y)$.
Pak selektor vybírá $y$ pro $\forall x$.

\begin{theorem}[O selektoru]\label{selector}
	Nechť $Q(x, y)$ je rekurzivně spočetný $($resp $Q(x_1, \ldots, x_n, y))$, pak $\exists \varphi \in$ CŘF:
	\[ \varphi(x) \downarrow \iff \exists y: Q(x, y) \]
		\[ \varphi(x) \downarrow \Rightarrow Q(x, \varphi(x)) \]
	Jinými slovy vybere $y$ pokud existuje.
\end{theorem}
\begin{proof}
	Pozor, nemůžeme vzít nejmenší, musíme vzít první protože lepší už třeba nebude.

	Q je r.s. $\Rightarrow$ má index $e$, napíšeme pomoci univerzálního predikátu
	\[ \exists s: T_2(e, x, y, s) \]
	Predikát zapíšeme jako množinu
	\[ dom(\varphi_e(x, y)) \]
	Pak pro dáne $x$ probereme všechny $\langle y, s \rangle$ a hledáme nejmenší dvojici tž platí
	\[ T_2(e, x, y, s) \]
	Neboli hledáme první $\langle y, s \rangle$ tž za $s$ kroků $\varphi_e(x,y) \downarrow$.

	Formálně:
	\[ \varphi(x) \simeq (\mu_{\langle y, s \rangle} T_2(e, x, y, s))_{2,1} \]
	indexy vydělí $y$.
\end{proof}

\begin{definition}
	Graf ČRF je
	\[ graph(\varphi) = \{ \langle x, y \rangle |\ \varphi(x) = y \} \]
\end{definition}
\begin{consequence}
	$\varphi$ je ČRF $\iff graph(\varphi)$ je r.s.
\end{consequence}
\begin{proof}
	"$\Rightarrow \langle x, y \rangle \in graph(\varphi) \Rightarrow \exists s$ (za $s$ kroků $\varphi(x) = y$).

	Což je r.s. predikát.

	"$\Leftarrow$" Aplikuj selektor. Volba v totalitním režimu, buď jeden kandidát nebo nic.
\end{proof}

\begin{note}
	Při zobecněních vyčíslitelnost do vyšších hierarchii, definujme vyčíslitelnost tak, že graf je rozumný.
\end{note}

\begin{theorem}[Postova podruhe]
	\[ Q(x,y) = (x \in M \land y = 1) \lor (x \in \overline{M} \land y = 0) \]

	Neboli $\varphi$ je charakteristická funkce množiny $M$.
\end{theorem}

\subsection{Imunní množiny}

\begin{definition}
	A je \emph{imunní} pokud je nekonečná a neobsahuje nekonečnou rekurzivní spočetnou podmnožinu.
	\[ W_x \subseteq A \Rightarrow |W_x| < \infty \]

	Je nekonečná, ale nemůžeme to efektivně zkontrolovat.
	Protože veškeré algoritmický zkontrolovatelné podmnožiny jsou konečné.
\end{definition}

\begin{definition}
	A je \emph{simple} pokud je rekurzivní spočetná a $\overline{A}$ je imunní.
\end{definition}

\begin{note}
	Postův problém: co je mezí Rekurzivní množiny a nerekurzivní množinou K?

	Definoval Simple, hypersimple, hyper-hyper ... atd až do Maximalní.
	Ale tato klasifikace neuspěla.
\end{note}

\begin{theorem}[Existence Simple]
	Existuje Simple množina.
\end{theorem}
\begin{proof}
	Uděláme predikát
	\[ Q(x, y) \iff y \in W_x \land y > 2x \]
	je rekurzivně spočetný protože nalezení je r.s. a druhá podmínka taky.
	Nechť $\varphi \in $ CŘF je selektor pro Q.
	Pak
	\[ A = range(\varphi) \]

	Podrobněji:
	\[ W_x \subseteq \overline{A} \Rightarrow W_x \subseteq \{ 0, \ldots, 2x \} \]
	Neboli $\overline{A}$ neobsahuje nekonečnou r.s. množinu.

	$\overline{A}$ nekonečná? \\
	Do $\{ 0, \ldots, 2x \}$ mohou přispět nejvýše $W_0, \ldots, W_{x - 1}$ množin.
	Neboli nejvýše $x$ čísel.
	Pak ale v $\overline{A}$ zůstane nejméně $(x + 1)$ čísel, neboli $\overline{A}$ je nekonečná.

	Dohromady $\overline{A}$ je imunní.

\end{proof}

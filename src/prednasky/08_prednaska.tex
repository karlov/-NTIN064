\section{\texorpdfstring{Limitní vyčíslitelnost}{Limitní vyčíslitelnost}}
\vspace{5mm}
\large

\subsection{Limitní vyčíslitelnost pro ORF}

\begin{definition}[Limitní vyčíslitelnost]
	A je \emph{limitně vyčíslitelná} právě když
	\[ \exists f \in ORF, \forall x: A(x) = \lim_s f(x, s) \]

	Taky je známá jako \emph{efektivní aproximace}.
\end{definition}

\begin{note}
	V diskretním prostoru, limita znamená že hodnota se stabilizuje (od určitého okamžiku je vždy 0 nebo 1).
\end{note}

\begin{theorem}[Limitní vyčíslitelnost]\label{lim_comp}
	\[ A \leq_T \empPr \iff A\ \text{je limitně vyčíslitelná} \]
	Věta je jednodušší verzi limitní vyčíslitelnosti, protože pro A máme všude definovanou charakteristickou funkce $C_A$.
\end{theorem}
\begin{proof}
	"$\Leftarrow$" Nechť
	\[ \exists f \in ORF, \forall x: A(x) = \lim_s f(x, s) \]
	Hledáme místo, od kterého se hodnota funkce nemění:
	\[ \mu_s(\forall j \geq s:(f(x, j) = f(x, y))) \]
	$\forall j \geq s: f(x, j) = f(x, s)$ je $\emptyset^{\prime}$-rekurzívní, protože negace je
	\[ \exists j \geq s:(f(x, j) \neq f(x, s)) \]
	Vnitřek je rekurzívní + existenční kvantifikátor dává r.s. (while cyklus).
	R.S. jsou $\geq_1 \empPr$, což implikuje taky $\geq_T \emptyset^{\prime}$.
	Pokud vrátíme negaci, tak je formule je r.s. (rekurzivita je uzavřená na negaci).

	Pak máme
	\[ \mu_s(\empPr-\text{rekurzívní podmínka}) \]
	Neboli je to $\empPr$-ČRF.
	Je to procedura, na kterou Halting problém umí odpovědět.
	Nejmenší $s$ hledáme ve while cyklu.

	Jelikož $A(x) = C_A(x)$ je všude definovaná, tak existuje příslušná limita:
	\[ \exists s \forall j \geq s:(f(x, j) = f(x, s)) \]
	Neboli je to dokonce $\empPr$-ORF, z čehož
	\[ A \leq_T \empPr \]

	"$\Rightarrow$" Předpokládáme $A \leq_T \empPr$, ekvivalentně:
	\[ \Phi_z(\empPr)(x) = A(x) \]
	Potřebujeme konvergentní posloupnost, která aproximuje $A$.
	\[ f(x, s) = \twopartdef {\Phi_{z, s}(\empPr_s)(x)}{\Phi_{z, s}(\empPr_s)(x) \downarrow}{s + 1}{\text{jinak}} \]
	Jelikož $\empPr$ je r.s., lze jí generovat po krocích.
	Takže $\empPr_s$ je $\empPr$ za $s$ kroků.
	Zřejmé:
	\[ \empPr = \bigcup \empPr_s \]

	Pak
	\[ A_s^{\prime} = W_{z_0, s}^A \]

	Z definice, $f \in ORF$.

	Ověříme existence limity.
	Pro dané $x$:
	\[ \Phi_{z}(\empPr)(x) = A(x) \downarrow \]
	Neboli musí existovat konečný začátek, pomoci kterého počítáme:
	\[ \exists \sigma \prec \empPr \Phi_{z}(\sigma)(x) = A(x)\]
	Taky počet kroků je konečný:
	\[ \exists s_1: \Phi_{z, s_1}(\sigma)(x) = A(x) \]

	$\sigma$ je konečný začátek $\empPr$
	\[ \sigma \prec \empPr \Rightarrow \forall j < |\sigma|: \sigma(j) = 1 \iff j \in \empPr \]
	Najdeme maximum $s_2$ takový, že
	\[ \sigma \prec \empPr_{s_2} \land \forall j \geq s_2 (\sigma \prec \empPr_j) \]
	Tedy $\sigma \prec \empPr$.

	Vezmeme $s:= \max(s_1, s_2)$, pak platí:
	\[ f(x, s) = A(x) \]
	$s_1$ nám stačí na generovaní $A(x)$, $s_2$ zaručí, že $\sigma$ je korektní aproximace $\empPr$.
\end{proof}

\begin{definition}[Modulus limity (P)]
	\[ m(x) = \mu_s(\forall j \leq x, \forall t \leq s: f(j, t) = f(j, s)) \]
\end{definition}

\begin{definition}[Weak Modulus(P)]
	Nebo taky \emph{first true moment}.
	Okamžik kdy aproximativní posloupnost se rovná $A(x)$.
	Ještě nezaručuje, že nadále bude stabilní.

	Až v moment úplného modulu.
\end{definition}

\begin{theorem}[Limitní vyčíslitelnost (P)]
	Nechť $g$ je všude definovaná funkce, A je r.s.
	\[ g \leq_T A \iff g = \lim_s(x, s) \]
	Kde \emph{modulus limity} $\leq_T A$.

	Podmínka se jmenuje \emph{definite stable}.
\end{theorem}
\begin{proof}
	"$\Leftarrow$" Pokud $m(x) \leq_T A \Rightarrow g \geq_T A$.
	Plyne z vlastnosti modulu.

	"$\Rightarrow$" Inspekce důkazu předchozí věty \cref{lim_comp}.

	V průběhu vypočtu se ptáme, jestli $\sigma$ je již korektní začátek $A$. Pokud ne, pokračujeme dál.
\end{proof}

\subsection{Obecná limitní vyčíslitelnost}

\begin{theorem}[Limitní vyčíslitelnost (P)]
	\begin{enumerate}
		\item Pokud $f \in ORF \Rightarrow \lim_s f(x, s)$ je $\empPr$-ČRF.
		\item Každá $\empPr$-ČRF je limitně vyčíslitelná:
			\[ \exists h \in ORF: \Phi_{z}(\empPr)(x) \simeq \lim_x h(z, x, s) \]
		Aproximace je \textbf{uniformní}.
	\end{enumerate}
\end{theorem}
\begin{proof}
	1) pokud $g(x) \simeq \lim_s f(x, s)$, tak jako v důkazu \cref{lim_comp}:
	\[ \mu_s(\forall j \geq s: f(x, j) = f(x, s)) \]
	Hledáme $s$ od kterého funkce se stabilizuje, a existuje limita.
	Pak
	\[ g(x) \simeq f(x, \mu_s(\ldots)) \]
	Jako i minule je to $\empPr$-ČRF.
	Problém ale je, že $s$ nemusí existovat, pak ale $g(x) \uparrow$.

	2) Je potřeba dokazovat opatrněji.
	\[ h_0(z, x, s) = \twopartdef { \langle \sigma, x, y \rangle}{ \sigma \prec \empPr_s \land \Phi_{z, s}(\sigma)(x) = y}{s + 1}{\text{jinak}} \]
	V předchozím důkazu bylo možné dávat přímo hodnotu výstupu a čekat pokud se stabilizuje.
	Nyní ale čekáme na stabilizaci celého aproximačního výpočtu.

	V průběhy výpočtu si ukládáme nejen výstup $y$ ale i jak jsme se k tomu dostali.
	Pak vydělíme $y$ pokud bude vyhovující:
	\[ h(z, x, s) = \twopartdef { (h_0(z, x, s))_{3, 3} }{ h_0(z, x, s) = h_0(z, x, s - 1)}{s + 1}{\text{jinak}} \]
	Q: proč je dostatečné zkontrolovat 2 body $s, s - 1$?

	Podobně jako v předchozím důkazu najdeme $s_1, s_2$:
	\[ \Phi_z(\empPr)(x) \downarrow \Rightarrow \exists s_2 \forall j \geq s_2: \sigma \prec \empPr_j \]

	Pak pro $s = \max(s_1, s_2)$ se stabilizuje celý výpočet.
	\[ h_0(z, x, s) = \langle \sigma, x, y \rangle \Rightarrow \lim_s h(z, x, s) = y \]

	2) Pokud existuje limita, tak z definice $h$ musí existovat i $\lim_s h_0(z, x, s)$.
	Pak
	\[ \lim_s h_0(z, x, s_0) = \langle \sigma, x, y \rangle \]
	Pak pro $s_0$ platí: $\forall j \geq s_0: \sigma \prec \empPr_j \Rightarrow \sigma \prec \empPr$.
	Dohromady:
	\[ \Phi_z(\empPr)(x) \downarrow \land \Phi_z(\empPr)(x) = y \land \sigma \prec \empPr \]

	Q: $(s + 1)$ je fiktivní hodnota, která rozhodí limitu?
\end{proof}

\begin{theorem}[Limitní vyčíslitelnost relativní (P)]
	\begin{enumerate}
		\item Pokud $f \in A-ORF \Rightarrow \lim_s f(x, s)$ je $A^{\prime}$-ČRF.
		\item Každá $A^{\prime}$-ČRF je limitně vyčíslitelná
			\[ \exists l, \forall s \forall z \forall x: \Phi_z(A^{\prime})(x) \simeq \Phi_l(A)(z, x, s) \]
			Na levé stráně je $A$-ČRF, na pravé je limita $A$-ORF.
			Aproximace je \textbf{uniformní}.
	\end{enumerate}
\end{theorem}
\begin{proof}
	Relativizáce předchozího důkazu.
\end{proof}

\begin{consequence}[Jump and lim]
	Jeden skok \cref{jump} odpovídá 1 limitnímu přechodu.
\end{consequence}

\begin{theorem}[Limitní vyčíslitelnost relativní nejobecnější (P)]
	Pokud $h \in ORF$:
	\[ \Phi_z(\emptyset^{(n + 1)})(x) \simeq \lim_{s_0} \ldots \lim_{s_n} h(z, x, s_0, \ldots, s_n) \]
\end{theorem}

\begin{note}[R.S. limitní hierarchie]
	\begin{itemize}
		\item Pokud $A$ je rekurzívní, tak $f(x, 0) = f(x, s) = A(x)$.
	Není potřeba aproximovat.
	\item $A$ je r.s. tak $f(x, s) = 1 \iff x \in A_s$. BUNO $f(x, 0) = 0$.
		Pak A je tzv 1-r.s., protože v každém sloupci dochází k nejvýš 1 změně.
	\item 2-r.s. (2 změny) bude rozdílem dvou r.s. množin.
	\item analogicky $n$-r.s. je booleovským rozdílem $n$ množin.
	\item $\omega$-r.s. nejvýš $\omega$ změn v každém sloupci.
		Ekvivalentně: $\exists h \in ORF:$ \# změn $\leq h(x)$.
		Obecně limita existuje, ale nejde udělat žadný odhad počtu změn ve sloupci.
	\end{itemize}
\end{note}
